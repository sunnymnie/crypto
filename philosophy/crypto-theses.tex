\documentclass{../notes}
\usepackage{hyperref}
\hypersetup{
    colorlinks=true,
    linkcolor=black,
    filecolor=magenta,
    urlcolor=blue,
    pdftitle={Overleaf Example},
    pdfpagemode=FullScreen,
    }
\title{My Crypto Theses}
\begin{document}
\maketitle

\tableofcontents


\part{Metaverse}

\section{Fan Tokens}

\section{Play to earn}

\section{Composable NFTs}

\section{NFT Financialization}

\section{Namespaces and Data sharing}

\part{Physically decentralized web}

\section{Physical network scaling}
\begin{itemize}
    \item Helium: very strong performer which bootstrapped hard hardware business model with right user economics
    \item Livepeer: decentralized video transcoding protocol gains traction due to rapidly growing video network
    \item Akash: stores docker containers at significatly reduced cost to cloud providers
    \item Andrena/Althea (pre-tokens) tackling internet service provider layer by enabling communities to set up hotspots and antennas that bring internet access to nearby towns
\end{itemize}

\section{Decentralized storage}
\begin{itemize}
    \item IPFS: decentralized system that allows any node to store data for a limited time
    \item Filecoin: incentivized storage network built on IPFS
    \item Arweave: long-term storage with own blockchain
    \item Sia: on-demand storage with own blockchain.
    \item Filebase/Pinata: decentralized storage aggregators
\end{itemize}

\part{DeFi 2.0}
\section{Algorithmic stablecoin}


\section{Emerging Protocol liquidity spectrum}
\begin{itemize}
    \item Olympus DAO
    \item Fei x Ondo Finance
    \item Tokemak
\end{itemize}


\section{Tokenized Funds}
\begin{itemize}
    \item Bankless \$BED and \$GMI
\end{itemize}


\section{Security and insurrance}
\begin{itemize}
    \item Smart contract insurance (Nexus--first crypto insurance unicorn, and definitely not last)
    \item Verified secure smart contract libraries and security-as-a-service. (Forta)
    \item Perpetuity on smart contract security researched. 
\end{itemize}

\part{ETH, Layers, Bridges}
\section{Liquid Staking}
\begin{itemize}
    \item Lido
    \item Anchor
\end{itemize}

\part{DAOs}
\section{DAO investor relations}
\begin{itemize}
    \item The graph: oh-chain data
\end{itemize}

\section{Test}
test \href{https://www.overleaf.com/learn/latex/Hyperlinks}{test}

\part{Working-theses}
\section{test}
test test \href{run:./messari-2022-crypto-theses.pdf}{test}
\textbf{BentoBox} - A new base-layer for future financial applications
that Sushi intends to offer. Simply described,
{\href{https://messari.io/article/sushi-bentobox-kashi-the-first-piece?referrer=asset:sushiswap}{BentoBox}}
is a vault that holds all assets deposited by users that can be utilized
by applications built on top of it. BentoBox aims to strengthen Sushi to
become the primary destination where users can interact with minimal gas
and maximal capital efficiency. BentoBox can facilitate multiple
transactions without having to execute multiple token approvals. Let's
imagine the user approved BentoBox access to the wallet's tokens once
and deposited them to the vault. The vault will allow those deposited
tokens to be used across any BentoBox applications, without needing to
approve the wallet's token access ever again. BentoBox optimizes for
capital efficiency by implementing a yield-generating strategy to idle
assets held in the vault. The strategy includes activities such as
lending, low-risk yield farming, staking, and flash loans.








\end{document}
