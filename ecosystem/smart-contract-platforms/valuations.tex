\documentclass{../../notes}
\title{Valuations - Smart Contract Platforms}
\begin{document}
\maketitle

\section{\href{run:./grow-the-pie--protocol-development-is-here-to-stay.pdf}{Grow the Pie: Protocol Development is Here to Stay}}
\begin{itemize}
    \item Value capture in web3: alignment of incentives between developers and users--positive feedback look. Most analysts focus on on-chain metrics (\# wallets, \# applications, flow of funds, trading volume, sentiment, etc), what is crucial to long-term success if code development. 
    \item Developer activity metrics: \href{https://www.coingecko.com/en/coins/solana\#developer}{CoinGecko} and \href{https://academy.santiment.net/metrics/development-activity/}{Santiment}. 
    \item Protocol development activity: all upward trend from 2018, ETH on league of its own with ~2000+ monthly code contributions, SOL ~400, DOT ~200+, ATOM ~200-. Year on year contributions: DOT picked up a lot more development activity compared to ATOM. DOT and ETH show significant uptrend in size of contributions year over year. 
    \item Protocol developer community: on the rise for each of the four protocols. ETH at ~250 monthly developers, rest with ~50, all on the rise from ~20 a year ago: talent migraiton from web2 to web3. 
    \item Ecosystem view: Sum all projects above 300 market-cap and remove cross-chain projects. \href{https://www.coingecko.com/en/categories/cosmos-ecosystem}{cosmos}: 15, \href{https://www.coingecko.com/en/categories/dot-ecosystem}{polkadot}: 9, \href{https://www.coingecko.com/en/categories/solana-ecosystem}{solana}: 13 as of Nov 1. 

    \item Ecosystem Development activity: Development activity of protocols built on top of L1. Cosmos in lead, but sustained levels for DOT and SOL. ATOM at ~3000 per month, DOT and SOL at ~2000. DOT has increase in average size of contributions in DOT ecosystem. 
    \item Ecosystem Developer Community: DOT and SOL attracting increasing number of developers given code in RUST. ATOM: ~225, DOT, SOL ~125. 
    \item Conclusion: Sustained code development is paramount for web3 protocols to maintain long-term and is healthy for all 4. 
\end{itemize}
\section{\href{run:./defining-l1-defi.pdf}{DeFining-L1-DeFi}}
\begin{itemize}
    \item Anticipated future economic activity is the core metric used to value smart contract platforms. Sum up all revenues from protocols on the network. ETH at \$400M with 60\% market share, followed by 20\%BSC, 8\% AVAX, 5\% SOL and MATIC, 3\% FTM
    \item Revenue depends on TVL: in 30 days span from mid-October to mid-Nov, ETH +7\%, BSC -10\%, AVAX +53\%, rest similar to ETH. ETH is losing marketshare in DeFi TVL while simultaneously growing. BSC is rapidly losing ground in TVL to other chains while not growing. AVAX fastest growing L1 in terms of TVL, the rest grew but either stayed in same \% of total TVL locked (SOL), or even lost market share (MATIC, FTM). 
    \item Defi growth rates: Though MATIC has low growth in terms of TVL, has made up with big growths in increase in DeFi users. AVAX still the shining star of the month
    \item Valuation Multiples: Eth valuation is average. SOL overvalued comapred to ETH, just means investors are pricing in future growth. FTM most discounted (as of Dec 16), then AVAX. MATIC valued equivalently to FTM and AVAX, even when AVAX and FTM are artifically boosting fundamentals with incentive programs. 
    \item Polygon Incentive Program: Polygon was highly overvalued, then incentive program, fundamentals grew quick, made Polygon undervalued, but when incentive program ended, Polygon was back to being rightly valued. Investors are thinking the same for AVAX and FTM
    \item Conclusion: BSC losing market share. Incentivizing growth has been key driver in AVAX and FTM. Watch out for new entrants: Terra and rest of Cosmos system and zk-rollups for Eth. Key metrics: DeFi users, revenue per TVL, diversification of revenue across protocols and sectors. 
\end{itemize}

\section{\href{run:./speculation-and-the-smart-contract-wars.pdf}{Speculation and the Smart Contract Wars}}
\begin{itemize}
    \item Valuation 101: Investors pricing future growth
    \item The End of the Beginning: Uncertainty invities opportunity and opportunity invites speculation: Eth competitors aggresively pursuing growth: heavy liquidity mining (try out platform), hosting hackathons, providing grant funding for builders, etc. Winner of this industry: potentially trillions of dollars. 
    \item Winner take all?: With analogy to current tech companies (AAPL, AMZN, MSFT, GOOGL, FB), seeminly multi trillion dollar companies doing similar things. 
    \item Ethereum as a magnet: Eth competitors will always be right behind Eth as long as smart contract sector is growing and there's trillions of dollars at stake
\end{itemize}




\end{document}

