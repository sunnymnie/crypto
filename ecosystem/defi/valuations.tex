\documentclass{../../notes}
\title{Valuations - DeFi}
\begin{document}
\maketitle

\section{\href{run:./defi-metrics-that-matter-for-a-price-recovery--fundamental-analysis.pdf}{DeFi Metrics that Matter for a Price Recovery: Fundamental Analysis}}
\begin{itemize}
    \item Fundamental analysis matters and there must be value accrual to protocol and investors, but price recovery can be more dependent to correlation to Bitcoin. Look at \# active users, trading volume, TVL, interest per year on lending protocols, amount deposited and oustanding loans for lending protocols
    \item Tokens that sold down the hardest during a crash seeminly make the biggest short-term recoveries. 
    \item High correlation between price and active user growth (seen in recovery phase). Idea: portfolio in DEXs with proportions weighted by active user growth
    \item Price performance of DEX also correlated to transaction volumes
    \item During bear markets, DEXs that allow staking for portion of fees outperforms short-term (SNX, CRV, KNC, SUSHI). During bull markets there is no correlation between protocols that allow fee accural to tokenholders (or generally long term)
    \item There is no correlation between TVL growth and price performance. Reason: capital is temporary and factors such as incentives or exploits can quickly change that. 
    \item Conclusion: During bear markets fundamentals matter. 
\end{itemize}




\end{document}

